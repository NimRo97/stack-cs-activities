\documentclass[12pt,a4paper]{article}
\usepackage[utf8]{inputenc}
\usepackage[margin=1cm]{geometry}
\usepackage[protrusion]{microtype}

\pagenumbering{gobble}

\newcommand{\solution}[3]{
\begin{itemize}
\item Purpose of the code: #1
\item Vulnerability: #2
\item Proposed fix: #3
\end{itemize}
}

\begin{document}

\noindent
Snippet 1:
\solution
{Read the username and possibly perform a privileged action.}
{Buffer overflow, since \texttt{gets()} does not check for buffer length.}
{Use \texttt{fgets()} and dynamically allocated memory.}

\noindent
Snippet 2:
\solution
{The \texttt{nresp} variable is set by the user. It stores the size of a packet to tell the server how many responses to expect. It's then used to allocate the \texttt{response} array and fill it with network data.}
{Integer overflow. On line 4, \texttt{nresp} is multiplied by the size of a cursor (4 bytes on a 32-bit architecture). Consequently, if users specify a value of \texttt{nresp} greater than 1073741823, the multiplication will exceed the maximum value that \texttt{unsigned int} can store (4294967295).}
{Check the size of the user input.}
% Source: OpenSSH before version 3.4

\noindent
Snippet 3:
\solution
{Copy two strings}
{Copying buffer data of the string \texttt{mechanism} with strcpy without checking size of the copy can result in buffer overflow.}
{Check the size of \texttt{mechanism} with \texttt{strlen()} before copying, or use \texttt{n}-bytes copy functions like \texttt{strncpy()}.}
% Source: University of Washington IMAP server, corrected in 1998

\noindent
Snippet 4:
\solution
{Reserve one byte for the null character at the end of the string when the size of \texttt{name} is equal to or greater than \texttt{npath}, and the last byte to be copied is “ (double quotation mark)}
{Index \texttt{i} increases again on line 9. This results in the null character being inserted one byte beyond the limit, generating an overflow.}
{Do not increment \texttt{i} on line 9.}
% Source: OpenBSD FTP demon

\noindent
Snippet 5:
\solution
{This code reads a packet header from the network and extracts a 32-bit length
field into the \texttt{length} variable. The \texttt{length} variable represents the total number of bytes in the packet, so the program first checks that the data portion of the packet isn't longer than 1024 bytes to prevent an overflow. It then tries to read the rest of the packet from the network by reading \texttt{(length - sizeof(struct header))} bytes into buffer. This makes sense, as the code wants to read in the packet's data portion, which is the total length minus the length of the header.}
{If users supply \texttt{length} less than \texttt{sizeof(struct header)}, the subtraction of \texttt{(length - sizeof(struct header))} causes an integer underflow and ends up passing a very large size parameter to \texttt{full\_read()}. This error could result in a buffer overflow because at that point, \texttt{read()} would essentially copy data into the buffer until the connection is closed, which would allow attackers to take control of the process.}
{Check also the lower bound of \texttt{length}.}
% Source: OpenBSD FTP demon

\end{document}